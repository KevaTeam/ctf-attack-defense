\documentclass[12pt]{eskdtext}
\usepackage[numbertop, numbercenter]{eskdplain}
\usepackage[utf8x]{inputenc}

% - Подключаем шрифты из пакета scalable-cyrfonts-tex
\usepackage{cyrtimes}

% - Отступ красной строки
\setlength{\parindent}{1.25cm}

% - Убирает точку в списке литературы
\makeatletter
\def\@biblabel#1{#1 }

% - Точки для всех пунктов в оглавлении
\renewcommand*{\l@section}{\@dottedtocline{1}{1.5em}{2.3em}}
\renewcommand*{\l@subsection}{\@dottedtocline{1}{1.5em}{2.3em}}
\renewcommand*{\l@subsubsection}{\@dottedtocline{1}{1.5em}{2.3em}}

% - Для переопределения списков
\renewcommand{\theenumi}{\arabic{enumi}}
\renewcommand{\labelenumi}{\theenumi)}
\makeatother

\usepackage{enumitem}
\setlist{nolistsep, itemsep=0.3cm,parsep=0pt}

% - ГОСТ списка литературы
\bibliographystyle{utf8gost705u}

% - Верикальные отступы заголовков 
\ESKDsectSkip{section}{1em}{1em}
\ESKDsectSkip{subsection}{1em}{1em}
\ESKDsectSkip{subsubsection}{1em}{1em}

% - Изменение заголовков
\usepackage{titlesec}
\titleformat{\section}{\centering\normalfont\normalsize}{\thesection}{1.0em}{}
\titleformat{\subsection}{\centering\normalfont\normalsize}{\thesubsection}{1.0em}{}
\titleformat{\subsubsection}{\centering\normalfont\normalsize}{\thesubsubsection}{1.0em}{}
\titleformat{\paragraph}{\centering\normalsize}{\theparagraph}{1.0em}{}

% - Оставим место под ТЗ 
%\setcounter{page}{4}

% - Для больших таблиц
\usepackage{longtable}
\usepackage{tabularx}
\renewcommand{\thetable}{\thesection.\arabic{table}}

% - Используем графику в документе
\usepackage{graphicx}
\graphicspath{{images/}}
\renewcommand{\thefigure}{\thesection.\arabic{figure}}

% - Счётчики
\usepackage{eskdtotal}

% - Выравнивание по ширине
\sloppy

% - Разрешить перенос двух последних букв слова
\righthyphenmin=2

\RequirePackage{enumitem}
\renewcommand{\alph}[1]{\asbuk{#1}}
\setlist{nolistsep}
\setitemize[1]{label=--, fullwidth, itemindent=\parindent, 
  listparindent=\parindent}% для дефисного списка
\setenumerate[1]{label=\arabic*), fullwidth, itemindent=\parindent, 
  listparindent=\parindent}% для нумерованного списка
\setenumerate[2]{label=\alph*), fullwidth, itemindent=\parindent, 
  listparindent=\parindent, leftmargin=\parindent}% для списка 2-ой ступени, который будет нумероваться а), б) и т.д.

% - Оформляем листинг кода (не использовать комментарии на русском!)
\usepackage{listings}  
\lstset{basicstyle=\ttfamily\small}
\lstset{extendedchars=\true}

% - выводим текст как есть с размером шрифта scriptsize
\makeatletter
\def\verbatim{\scriptsize\@verbatim \frenchspacing\@vobeyspaces \@xverbatim}
\makeatother


\begin{document}
 \newpage
\ESKDthisStyle{empty}

\begin{center}
 Министерство образования и науки Российской Федерации\\
 Федеральное государственное бюджетное образовательное учреждение высшего профессионального образования\\
 <<ТОМСКИЙ ГОСУДАРСТВЕННЫЙ УНИВЕРСИТЕТ СИСТЕМ УПРАВЛЕНИЯ И РАДИОЭЛЕКТРОНИКИ>> (ТУСУР)\\
 Кафедра комплексной информационной безопасности электронно-вычислительных систем (КИБЭВС)\\
\end{center}


\begin{center}
Курсовой проект по дисциплине \\
«Технологии и методы программирования» \\
ПРОГРАММНЫЙ КОМПЛЕКС ДЛЯ ПРОВЕДЕНИЯ СОРЕВНОВАНИЙ В ОБЛАСТИ ИНФОРМАЦИОННОЙ БЕЗОПАСНОСТИ
\end{center}
\vfill

\begin{flushleft}
\begin{minipage}{0.45\textwidth}
 \begin{flushright}
  УТВЕРЖДАЮ\\
  заведующий каф. КИБЭВС, ректор
  \underline{\hspace{3cm}}А.А. Шелупанов \\
  <<\underline{\hspace{1cm}}>>\underline{\hspace{3cm}}2015г.\\
 \end{flushright}
\end{minipage}
\end{flushleft}

\vfill

\vfill
\begin{flushright}
\begin{minipage}{0.45\textwidth}
 \begin{flushleft}
  Выполнили \\
  студенты гр. 723 \\
  \underline{\hspace{3cm}}Д.Е. Муковкин \\
  \underline{\hspace{3cm}}Г.В. Засыпкин \\
  <<\underline{\hspace{1cm}}>>\underline{\hspace{3cm}}2015г.\\
 \end{flushleft}
\end{minipage}
\end{flushright}

\vfill

\begin{flushright}
\begin{minipage}{0.45\textwidth}
 \begin{flushleft}
  СОГЛАСОВАНО \\
  доцент каф. КИБЭВС \\
  \underline{\hspace{3cm}}В.А. Ефремов \\
  <<\underline{\hspace{1cm}}>>\underline{\hspace{3cm}}2015г.\\
 \end{flushleft}
\end{minipage}
\end{flushright}

\vfill

\begin{center}
 2015
\end{center}

 \newpage
\ESKDthisStyle{empty}
\paragraph{\hfill РЕФЕРАТ \hfill}
Курсовая работа содержит \ESKDtotal{page} страниц, \ESKDtotal{figure} рисунка, \ESKDtotal{table} таблицы, \ESKDtotal{bibitem} источников, \ESKDtotal{appendix} приложение.

%допилить ключевые слова
КОМПЬЮТЕРНАЯ ЭКСПЕРТИЗА, ФОРЕНЗИКА, ЛОГИ, QT, XML, GIT, LATEX, MOZILLA THUNDERBIRD, MS OUTLOOK, WINDOWS, PST, MSG, RTF, HTML, БИБЛИОТЕКИ, РЕПОЗИТОРИЙ, ПОЧТОВЫЙ КЛИЕНТ, SQLLITE, РЕЕСТР, МЕТА-ДАННЫЕ, READPST, JPEG, PNG, ID3V1, JFIF, RIFF, CHUNK, DBX, C++, DOXYGEN.

Цель работы --- создание программного комплекса, предназначенного для проведения компьютерной экспертизы.

Задачей, поставленной на данный семестр, стало написание программного комплекса, имеющего следующие возможности: 
\begin{enumerate}
\item сбор и анализ информации из журналов истории браузеров;
\item сбор и анализ информации из мессенджеров;
\item сбор и анализ информации из почтовых приложений;
\item поиск медиа-файлов (аудио, видео, изображение) и извлечение мета-данных из них;
\item сбора информации об установленном ПО по остаточным файлам.
\item сбор и анализ информации из реестра Windows.
\end{enumerate}

Результаты работы в данном семестре:

\begin{itemize}
\item реализован импорт истории посещений, поисковых запросов, загруженных файлов, списка установленных расширений, информации о версии программы и подключенном аккаунтеGoogle из приложения Google Chrome;
\item реализован алгоритм сканирования директорий ОС Windows на содержание файлов, оставшихся при установке или после удаления различных программ;
\item реализован алгоритм извлечения адресата, отправителя, темы, даты и текста
сообщения из файлов формата <<.dbx>>, используемого MS Outlook для хранения сообщений;
\item реализован алгоритм извлечения времени, даты, отправителя, получателя и текст
сообщения почтового клиента Mozilla Thunderbird;
\item реализован алгоритм извлечения мета-данных из фалов формата ID3v1, JFIF и RIFF;
\item реализован алгоритм извлечения информации из реестра Windows (.reg-файлов).
\end{itemize}

Пояснительная записка выполнена при помощи системы компьютерной вёрстки \LaTeX.

 
 \newpage
 \ESKDthisStyle{empty}
 \section*{Список исполнителей}

Муковкин Д.Е. -- программист, ответственный исполнитель.

Задачи:
\begin{itemize}
\item выбор языка программирования;
\item создание архитектуры ядра;
\item развертывание инфраструктуры;
\item разработка модуля проверки сервисов, сдача флагов;
\item настройка и тестирование платформы.
\end{itemize}

Койшинов Т.С. -- программист.

Задачи:
\begin{itemize}
\item разработка структуры БД в API;
\item доработка панели администратора;
\item разработка API функций;
\item разработка модуля таблицы результатов.
\end{itemize}
 
 % - содержание
 \newpage
 \ESKDstyle{plain}
 \tableofcontents

 \newpage
 \ESKDstyle{plain}
 \section*{Введение}
 \addcontentsline{toc}{section}{Введение}
 CTF (Capture the flag, Захват флага) --- это командные соревнования, целью которых является оценка умения участников атаковать и защищать компьютерные системы. По типу, соревнования делятся на два типа: task-based (квесты), attack-defense (классические соревнования).

Для проведения соревнований типа attack-defense, каждой команде выдается сервер, на котором имеется ряд сервисов, одинаковые у всех. Сервисами являются программы, которые должны быть запущены на протяжении всего времени соревнований. Раз в минуту жюрейская система проверяет работу сервисов, присылает новый флаг, а также проверяет его наличие на сервере. В роли флага выступает случайно сгенерированная строка, определенной длины. Сервисы заведомо имеют уязвимости, через которые можно украсть флаг. Команда должна найти уязвимости в сервисах и закрыть их. Но также она должна эксплуатировать эти уязвимости на сервисах других команд, похищая флаги. 

Команда Keva имеет большой опыт в проведении соревнований CTF. Во время проведения межрегиональных межвузовских соревнований в области информационной безопасности SibirCTF 2014, 2015 использовались наработки Екатеринбургской команды HackerDom. Однако их решение не подходит нам по некоторым критически важным параметрам. Поэтому  было принято решение написать собственную платформу для проведения соревнований CTF.

Платформа должна отвечать следующим требованиям:
\begin{itemize}
\item Низкое потребление ресурсов сервера;
\item Управление игрой через графический интерфейс;
\item Возможность горизонтального масштабирования.
\end{itemize}

Новая платформа позволит нам расширить формат проведения соревнований, избавиться от зависимости в использовании чужих программных продуктов, а также решать непредвиденные проблемы в кратчайшие сроки.

При разработке платформы должны быть использованы уже имеющиеся наработки, в частности, панель администратора и API для task-based проведения соревнований. 

 \newpage
 \section{Назначение и область применения}
Разрабатываемый программно-аппаратный комплекс предназначен для проведения игр в области информационной безопасности Capture the flag по типу Attack-Defense.
\section{Постановка задачи}
\setcounter{figure}{0}
На данный семестр были поставлены следующие задачи:

\begin{itemize}
\item определение индивидуальных задач для каждого участника проектной группы;
\item исследование предметных областей в рамках индивидуальных задач;
\item доработка программных модулей.
\end{itemize}

Задачи, связанные с разработкой программного комплекса:

\begin{enumerate}
\item доработка ядра платформы;
\item доработка модуля проверяющей системы;
\item доработка модуля сдачи флага;
\item доработка модуля рейтинговой таблицы;
\item доработка модуля администраторской панели.
\end{enumerate}

Задачи, связанные с развитием экосистемы:

\begin{enumerate}
\item создание сервисов;
\item создание руководства пользователя;
\item создание руководства администратора;
\item продвижение платформы в CTF сообществе.
\end{enumerate}

Задачи, связанные с тестированием программного комплекса:

\begin{enumerate}
\item проверка работы ядра;
\item проверка работы модуля проверяющей системы;
\item проверка работы модуля сдачи флага;
\item проверка работы модуля рейтинговой таблицы;
\item проверка работы модуля администраторской панели;
\item проверка работы всех модулей с помощью нагрузочного тестирования.
\end{enumerate}

\section{Инструменты}
\setcounter{figure}{0}
\subsection{Система контроля версий Git}
Для разработки программного комплекса для проведения соревнований в области информационной безопасности было решено использовать Git.

Git --- распределённая система управления версиями файлов. Проект был создан Линусом Торвальдсом для управления разработкой ядра Linux  как противоположность  системе управления версиями Subversion (также известная как «SVN») \cite{progit}.

При работе над одним проектом команде разработчикоа необходим инструмент для совместного написания, резервного копирования и тестирования программного обеспечения. Используя Git, мы имеем:
\begin{itemize}
\item возможность удаленной работы с исходными кодами;
\item возможность создавать свои ветки, не мешая при этом другим разработчикам;
\item доступ к последним изменениям в коде, т.к. все исходники хранятся на сервере git.keva.su;
\item исходные коды защищены, доступ к ним можно получить лишь имея RSA-ключ;
\item возможность откатиться к любой стабильной стадии проекта.
\end{itemize}

Основные постулаты работы с кодом в системе Git:

\begin{itemize}
\item каждая задача решается в своей ветке;
\item необходимо делать коммит как только был получен осмысленный результат;
\item ветка master мержится не разработчиком, а вторым человеком, который производит вычитку и тестирование изменения;
\item все коммиты должны быть осмысленно подписаны/прокомментированы.
\end{itemize}

Для работы над проектом был поднят собственный репозиторий на сервере gitlab2.keva.su.

Исходные файлы проекта:

git@gitlab2.keva.su:sibirctf/sibirctf-attack-defense.git

\subsection{Система компьютерной вёрстки \TeX}
\TeX\ --- это созданная американским математиком и программистом Дональдом Кнутом система для вёрстки текстов. Сам по себе \TeX\ представляет собой специализированный язык программирования.Каждая издательская система представляет собой пакет макроопределений этого языка.

\LaTeX\ --- это созданная Лэсли Лэмпортом издательская система на базе \TeX'а\cite{lvovskyi}. \LaTeX\ позволяет пользователю сконцентрировать свои услия на содержании и структуре текста, не заботясь о деталях его оформления.

Для подготовки отчётной и иной документации нами был выбран \LaTeX\, так как совместно с системой контроля версий Git он предоставляет возможность совместного создания и редактирования документов. Огромным достоинством системы \LaTeX\ то, что создаваемые с её помощью файлы обладают высокой степенью переносимости \cite{latexrus}.

Совместно с \LaTeX\ часто используется Bib\TeX\ --- программное обеспечение для создания форматированных списков библиографии. Оно входит в состав дистрибутива \LaTeX\ и позволяет создавать удобную, универсальную и долговечную библиографию. Bib\TeX\ стал одной из причин, по которой нами был выбран \LaTeX\ для создания документации.

\subsection{Язык программирования Python}
Python --- высокоуровневый язык программирования общего назначения, ориентированный на повышение производительности разработчика и читаемости кода. Синтаксис ядра Python минималистичен. В то же время стандартная библиотека включает большой объём полезных функций \cite{python_wiki}. 


Основные архитектурные черты — динамическая типизация, автоматическое управление памятью, полная интроспекция, механизм обработки исключений, поддержка многопоточных вычислений и удобные высокоуровневые структуры данных. Код в Python организовывается в функции и классы, которые могут объединяться в модули.

Основными преимуществами языка программирования Python являются большое количество библиотек, кроссплатформенность, широкие возможности профилирования кода. 

Язык обладает чётким и последовательным синтаксисом, продуманной модульностью и масштабируемостью, благодаря чему исходный код написанных на Python программ легко читаем. При передаче аргументов в функции Python использует вызов по соиспользованию.

Разработка языка Python была начата в конце 1980-х годов[10] сотрудником голландского института CWI Гвидо ван Россумом.
\subsection{База данных MongoDB}
Для реализации программного комплекса для проведения соревнований в области информационной безопасноти была использована база данных MongoDB.

MongoDB --- документо-ориентированная система управления базами данных (СУБД) с открытым исходным кодом, не требующая описания схемы таблиц. Написана на языке C++. \cite{progit}.

Основные возможности MongoDB:
\begin{itemize}
\item документо-ориентированное хранение (JSON-подобная схема данных);
\item достаточно гибкий язык для формирования запросов;
\item динамические запросы;
\item поддержка индексов;
\item профилирование запросов;
\item быстрые обновления «на месте»;
\item эффективное хранение двоичных данных больших объёмов, например, фото и видео;
\item журналирование операций, модифицирующих данные в базе данных
\item поддержка отказоустойчивости и масштабируемости: асинхронная репликация, набор реплик и распределения базы данных на узлы;
\item может работать в соответствии с парадигмой MapReduce;
\item полнотекстовый поиск, в том числе на русском языке, с поддержкой морфологии.
\end{itemize}

Архитектура:

СУБД управляет наборами JSON-подобных документов, хранимых в двоичном виде в формате BSON. Хранение и поиск файлов в MongoDB происходит благодаря вызовам протокола GridFS. Подобно другим документо-ориентированным СУБД (CouchDB и др.), MongoDB не является реляционной СУБД. В СУБД:

\begin{itemize}
\item нет такого понятия, как «транзакция». Атомарность гарантируется только на уровне целого документа, то есть частичного обновления документа произойти не может;
\item Отсутствует понятие «изоляции». Любые данные, которые считываются одним клиентом, могут параллельно изменяться другим клиентом.
\end{itemize}

В MongoDB реализована асинхронная репликация в конфигурации «ведущий — ведомый» (англ. master — slave), основанная на передаче журнала изменений с ведущего узла на ведомые. Поддерживается автоматическое восстановление в случае выхода из строя ведущего узла. Серверы с запущенным процессом mongod должны образовать кворум, чтобы произошло автоматическое определение нового ведущего узла. Таким образом, если не используется специальный процесс-арбитр (процесс mongod, только участвующий в установке кворума, но не хранящий никаких данных), количество запущенных реплик должно быть нечётным.

\subsection{TrackingTime - Система управления проектами и задачами}

TrackingTime --- это сервис, предоставляющий возможность управлять своими проектами, задачами, персоналом.

Основным преимуществом является удобный учет времени ответственного за задачу. 
Проект декомпозируется на этапы и задачи.
Каждая задача назначается на ответственного работника, который в свою очередь при её выполнении устанавливает таймер.

Приложение доступно на всех основных платформах (Windows, Linux, OS X, iOS, Android).

\subsection{Flask - микрофреймворк для Python}
\ESKDthisStyle{empty}
Flask - это микрофреймворк, написанный на языке программирования Python, основанный на Werkzeug и Jinja 2. Выпускается под BSD лицензией.

Слово «микро» означает, что цель написания фреймворка - сохранить ядро простым, но в то же время легко расширяемой. По умолчанию Flask не включает в себя уровень абстракции базы данных, валидацию форм или другие библиотеки, которые с легкостью можно подключить. Вместо этого Flask поддерживает механизм расширения существующего кода так, как будто он уже был подключен к нему. Множество расширений предоставляют интеграцию с базой данных, валидацию форм, поддержку загрузки файлов, аутентификации пользователя и другие полезные функции.

Особенности фреймворка 
\begin{itemize}
\item Удобная работа с URL;
\item Богатые возможности шаблонизатора;
\item Минимальные требования к ресурсам в сравнении с аналогичными фреймворками.
\end{itemize}

Flask используется в модуле таблицы рейтинга. 

\section{Технические характеристики}
\subsection {Требования к аппаратному обеспечению}

Необходимо не менее двух компьютеров, находящихся в локальной сети (компьютеры команд-участников), а также сервер, на котором запускается платформа.

Минимальные системные требования для сервера:

\begin{itemize}
\item процессор 1ГГц Pentium 4;
\item оперативная память 512 Мб;
\item место на жёстком диске -- 9 Гб.
\end{itemize}

Минимальные системные требования для компьютеров команд-участников определяются исходя из написанных сервисов.

\subsection {Требования к программному обеспечению}
Для корректной работы разрабатываемого программного комплекса сервер должен работать под управлением ОС Ubuntu Linux. 
Также должны быть установлены следующие пакеты: Python 3.X, MongoDB, PHP 5, MySQL, Веб-сервер Apache 2 или Nginx.


Требования к ПО команд-участников выставляются уже при непосредственном проведении соревнований

\subsection {Требования к сервисам}
\subsection{Выбор единого формата выходных файлов}
Для вывода результата был выбран формат XML-документов, так как с данным форматом лего работать при помощи программ, а результат работы данного комплекса в дальнейшем планируется обрабатывать при помощи программ.

XML - eXtensible Markup Language или расширяемый язык разметки. Язык XML представляет собой простой и гибкий текстовый формат, подходящий в качестве основы для создания новых языков разметки, которые могут использоваться в публикации документов и обмене данными \cite{xml}. Задумка языка в том, что он позволяет дополнять данные метаданными, которые разделяют документ на объекты с атрибутами. Это позволяет упростить программную обработку документов, так как структурирует информацию.

Простейший XML-документ может выглядеть так:


\begin{verbatim}
<?xml version="1.0"?>
<list_of_items>
<item id="1"\><first/>Первый</item\>
<item id="2"\>Второй <subsub_item\>подпункт 1</subsub_item\></item\>
<item id="3"\>Третий</item\>
<item id="4"\><last/\>Последний</item\>
</list_of_items>
\end{verbatim}


Первая строка - это объявление начала XML-документа, дальше идут элементы документа <list\_of\_items> - тег описывающий начало элемента \\list\_of\_items, </list\_of\_items> - тег конца элемента. Между этими тегами заключается описание элемента, которое может содержать текстовую информацию или другие элементы (как в нашем примере). Внутри тега начала элемента так же могут указывать атрибуты элемента, как например атрибут id элемента item, атрибуту должно быть присвоено определенное значение.


\section{Разработка программного обеспечения}
\setcounter{figure}{0}
 
\subsection{Архитектура}
\subsubsection{Основной алгоритм}
В ходе разарботки был применен видоизменнённый шаблон проектирования Factory method.


\subsection{Модуль: прием и проверка принятых флагов} % - Отчёт flags
В соревнованиях по информационной безопасности задача команд найти уязвимость и эксплуатировать её, добывая секретную информацию, в нашем случае флаги. Целью модуля является прием и проверка на валидность флагов.

\subsubsection{Принцип работы}

Программа реализована с использованием вебсокетов. На порту, полученном с API, программа сравнивает IP адрес клиента с данными о IP адресах клиента или его подсети в базе данных и при нахождении его, клиент определяется как одна из команд и может отправить серверу строку. Строка проверяется на длину символов. Так же флаг проверяется на наличие в базе данных, времени его жизни (флаги валидны определенное количество времени), принадлежность другой команде (свои флаги сдавать нельзя) и статус сервиса (аналогичный сервис сдающей команды должен быть поднят). 

Ниже представлен алгоритм работы flags.py (Рисунок 5.1)

\begin{figure}[ht!]
\center{\includegraphics[width=0.4\linewidth]{individual_reports/GPO5.png}}
\caption{Алгоритм модуля flags.py}
\end{figure} 

\clearpage


\clearpage


\newpage
\subsection{Модуль: прием и проверка принятых флагов} % - Отчёт flags
В соревнованиях по информационной безопасности задача команд найти уязвимость и эксплуатировать её, добывая секретную информацию, в нашем случае флаги. Целью модуля является прием и проверка на валидность флагов.

\subsubsection{Принцип работы}

Программа реализована с использованием вебсокетов. На порту, полученном с API, программа сравнивает IP адрес клиента с данными о IP адресах клиента или его подсети в базе данных и при нахождении его, клиент определяется как одна из команд и может отправить серверу строку. Строка проверяется на длину символов. Так же флаг проверяется на наличие в базе данных, времени его жизни (флаги валидны определенное количество времени), принадлежность другой команде (свои флаги сдавать нельзя) и статус сервиса (аналогичный сервис сдающей команды должен быть поднят). 

Ниже представлен алгоритм работы flags.py (Рисунок 5.1)

\begin{figure}[ht!]
\center{\includegraphics[width=0.4\linewidth]{individual_reports/GPO5.png}}
\caption{Алгоритм модуля flags.py}
\end{figure} 

\clearpage


\newpage
\section*{Заключение}
\addcontentsline{toc}{section}{Заключение}
В данном семестре нашей группой была выполнена часть работы по созданию автоматизированного программного комплекса для проведения компьютерной экспертизы, проанализированы дальнейшие перспективы и поставлены цели для дальнейшего развития проекта.
 
 
 \newpage
 \renewcommand{\refname}{Список использованных источников}
 \bibliography{lit}

 \ESKDappendix{Обязательное}{\normalfont Компакт-диск}
 Компакт-диск содержит: 
 \begin{itemize}
 \item электронную версию пояснительной записки в форматах *.tex и *.pdf;
 \item актуальную версию программного комплекса для проведения компьютерной экспертизы;
 \item тестовые данные для работы с программным комплексом;
 \item документацию к проекту в html-формате.
 \end{itemize}
 
\end{document}
