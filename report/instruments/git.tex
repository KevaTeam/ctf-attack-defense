Для разработки программного комплекса для проведения соревнований в области информационной безопасности было решено использовать Git.

Git --- распределённая система управления версиями файлов. Проект был создан Линусом Торвальдсом для управления разработкой ядра Linux  как противоположность  системе управления версиями Subversion (также известная как «SVN») \cite{progit}.

При работе над одним проектом команде разработчикоа необходим инструмент для совместного написания, резервного копирования и тестирования программного обеспечения. Используя Git, мы имеем:
\begin{itemize}
\item Возможность удаленной работы с исходными кодами;
\item Возможность создавать свои ветки, не мешая при этом другим разработчикам;
\item Доступ к последним изменениям в коде, т.к. все исходники хранятся на сервере git.keva.su;
\item Исходные коды защищены, доступ к ним можно получить лишь имея RSA-ключ;
\item Возможность откатиться к любой стабильной стадии проекта.
\end{itemize}

Основные постулаты работы с кодом в системе Git:

\begin{itemize}
\item Каждая задача решается в своей ветке;
\item Необходимо делать коммит как только был получен осмысленный результат;
\item Ветка master мержится не разработчиком, а вторым человеком, который производит вычитку и тестирование изменения;
\item Все коммиты должны быть осмысленно подписаны/прокомментированы.
\end{itemize}

Для работы над проектом был поднят собственный репозиторий на сервере gitlab2.keva.su.

Исходные файлы проекта:

git@gitlab2.keva.su:sibirctf/sibirctf-attack-defense.git
