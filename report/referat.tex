\newpage
\ESKDthisStyle{empty}
\paragraph{\hfill РЕФЕРАТ \hfill}
Курсовая работа содержит \ESKDtotal{page} страниц, \ESKDtotal{figure} рисунка, \ESKDtotal{table} таблицы, \ESKDtotal{bibitem} источников, \ESKDtotal{appendix} приложение.

%допилить ключевые слова
КОМПЬЮТЕРНАЯ ЭКСПЕРТИЗА, ФОРЕНЗИКА, ЛОГИ, QT, XML, GIT, LATEX, MOZILLA THUNDERBIRD, MS OUTLOOK, WINDOWS, PST, MSG, RTF, HTML, БИБЛИОТЕКИ, РЕПОЗИТОРИЙ, ПОЧТОВЫЙ КЛИЕНТ, SQLLITE, РЕЕСТР, МЕТА-ДАННЫЕ, READPST, JPEG, PNG, ID3V1, JFIF, RIFF, CHUNK, DBX, C++, DOXYGEN.

Цель работы --- создание программного комплекса, предназначенного для проведения компьютерной экспертизы.

Задачей, поставленной на данный семестр, стало написание программного комплекса, имеющего следующие возможности: 
\begin{enumerate}
\item сбор и анализ информации из журналов истории браузеров;
\item сбор и анализ информации из мессенджеров;
\item сбор и анализ информации из почтовых приложений;
\item поиск медиа-файлов (аудио, видео, изображение) и извлечение мета-данных из них;
\item сбора информации об установленном ПО по остаточным файлам.
\item сбор и анализ информации из реестра Windows.
\end{enumerate}

Результаты работы в данном семестре:

\begin{itemize}
\item реализован импорт истории посещений, поисковых запросов, загруженных файлов, списка установленных расширений, информации о версии программы и подключенном аккаунтеGoogle из приложения Google Chrome;
\item реализован алгоритм сканирования директорий ОС Windows на содержание файлов, оставшихся при установке или после удаления различных программ;
\item реализован алгоритм извлечения адресата, отправителя, темы, даты и текста
сообщения из файлов формата <<.dbx>>, используемого MS Outlook для хранения сообщений;
\item реализован алгоритм извлечения времени, даты, отправителя, получателя и текст
сообщения почтового клиента Mozilla Thunderbird;
\item реализован алгоритм извлечения мета-данных из фалов формата ID3v1, JFIF и RIFF;
\item реализован алгоритм извлечения информации из реестра Windows (.reg-файлов).
\end{itemize}

Пояснительная записка выполнена при помощи системы компьютерной вёрстки \LaTeX.
